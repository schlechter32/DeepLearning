\mainsection{6}{Overfitting and regularization}{25/05/2020}
\section{Model capacity and overfitting / underfitting}
\putfigure{1.0}{1}{0.4}{Images/ModelFitting}{Model fitting curve}
\section{Weight norm penalty}
change on cost function: \\
\textbullet old cost function: $ L(\underline{\theta}) $\\
\textbullet New regularized cost function: $ L_r (\underline{\theta}) = L /\underline{\theta}  + \sum_{l=1}^{L} \lambda_i P(\W_L)$ \\
$  \lambda \geq 0: $ regularized parameters \\
$  P(\W_L) \geq 0 :  $ penalty terms, penalize $  \underline{\theta} $ with large $ P(\W_L) $\\
$ \rightarrow  $ A compromise between $ \underset{\underline{\theta}}{min} L(\underline{\theta}) $ and the min $ \underset{\underline{\theta}}{min} P(\W_L) $\\
$ \lam_L   $ determine the relative contributions of $  L(\underline{\theta}), P(\W_L) $. \\
$  \lam_L = 0, \forall L  $ : no regularization \\
common choice of $  P(\W_L) $: \\
\textbullet a) l2-regularization: we use $ l_2-norm $ of $ vec(\W_L) $ \\
$  P(\W_L) = || vec (\W_L ) ||^2_{2} = \sum_{i}^{} \sum_{j}^{} W_{L,ij}^2 \rightarrow $ prefer $ \underline{\theta} $ with small weight energy \\
$\rightarrow$ better generalization, see E6.1\\
Bias vector $ \b_l: $ no amplification effext of $ \x_{l-1} \rightarrow  $ no need for regularization \\
\textbf{Mathematical analysis:}\\
$  L_r (\underline{\theta}  ) = L(\underline{\theta}) + \lam || \underline{\theta} ||^2 $ for simplicity \\
$ \underline{\nabla} L_r (\underline{\theta}) = \underline{\nabla} L(\underline{\theta}) + 2 \cdot \lam \underline{\theta} $ \\
$ \underline{\theta}^{t+1} = \underline{\theta}^t - \gamma^t \underline{\nabla } L_r (\underline{\theta}^t) $ \\
 $\underbrace{(1- 2 \lam \gamma^t)}_{0 << factor <1}  \underline{\theta}^t - \gamma^t \underline{\nabla} L(\underline{\theta}^t) $\\
 l2-regularization leads to \textbf{weight decay } \\
\textbf{ b) l1-regularization: } use l1-norm of $ vec(\W_L) $ \\
$  P(\W_L) = || vec (\W_L) ||_1 = \sum_i \sum_j | W_{L,ij} | \\
\rightarrow  $ prefer sparse $ \W_L $ with many zero elements\\
\section{Early stopping}
change on optimizer:\\
\putfigure{1.0}{1}{0.2}{Images/StopTraining}{Early stopping} 
\section{Data augmentation} 
change on dataset, training with $ \infty $ many training data $\rightarrow$ no over fitting, in practice, training set is limited in size\\
\textbf{data augmentation:} Generation of artificial realistic training samples: 
\section{Ensamble learning}
change dataset / model / cost function /  optimizer\\
\putfigure{1.0}{1}{0.4}{Images/EnsembleLearning}{Ensamble learning explained} 
\section{Dropout}
change on model \\
An implicit ensemble method\\
\putfigure{1.0}{1}{0.4}{Images/DropOut}{Dropout explained} 
\section{Hyperparameter optimization}
\putfigure{1.0}{1}{0.4}{Images/HyperParametersExplained}{Hyperparameters explained} \\
Solution for b): Use validation data set D\\
up to now separating dataset in 2 for training and test, but now split in three parts training set $ D_{train} $ (large). test set $ D_{test} $(small) and validation set $ D_{val} $ (small, same size as test)\\
\putfigure{1.0}{1}{0.4}{Images/TrainingTestValidationSet}{Summary of data sets} \\
\textbf{Training:} $ \underline{\theta} $ and hyperparameter and $ \underline{\eta} $\\
for $ \underline{\eta} = ..... $ learn $ \underline{\theta} $ of $ f(\x ; \underline{\theta} ; \underline{\eta}) $ frpm $  D_{train} $\\
calculate validation error $ (\underline{\eta}) $ on $ D_{val} $\\
min validation error ($ \underline{\eta} $)\\
calculate test error of $ f(\x ; \underline{\theta} ; \underline{\eta}) $ on $ D_{test} $\\
\putfigure{1.0}{1}{0.4}{Images/OptimizeHyperparameters}{Approaches to optimize hyperparameters} \\

















